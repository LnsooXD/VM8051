\documentclass{article}
\title{VM8051 User Guide}
\author{Luk Bettale}
\date{}

\begin{document}

\maketitle

\section*{Introduction}
VM8051 is a virtual machine that simulates a 8051 CPU. The software is
composed of a library \texttt{lib8051}, and an interactive program
that allows to execute a 8051 program step by step.

\section*{Using \texttt{vm8051}}
VM8051 can be launched by command line. The binary input file has to
be specified as first argument. The \texttt{-m} option instructs the
VM to show only minimal information about the state (only registers
and CPU related SFRs).

\begin{verbatim}
vm8051 [-m] input
\end{verbatim}

\section*{Main commands}
The following commands are available:

\begin{tabular}{ll}
  \texttt{s}&\textbf{S}tep instruction (step into call/jump)\\
  \texttt{q}&\textbf{Q}uit\\
  \texttt{p}&\textbf{P}rint state (do nothing)\\
  \texttt{r}&\textbf{R}eset virtual machine\\
  \texttt{b} \textit{address}&\textbf{B}reakpoint at \textit{address}\\
  \texttt{d} \textit{address}&\textbf{D}elete breakpoint at \textit{address}\\
  \texttt{j} \textit{address}&\textbf{J}ump to \textit{address}\\
  \texttt{k}&\textbf{K}ill (skip) next instruction\\
  \texttt{n}&\textbf{N}ext instruction in the code (skip call/jump)\\
  \texttt{g} \textit{address}&\textbf{G}o to \textit{address} (run
  code), stop at breakpoints\\
  \texttt{c}&\textbf{C}ontinue (run code), stop at breakpoints\\
  \texttt{w} \textit{ncy}&\textbf{W}ait \textit{ncy} cycles (run code),
  stop at breakpoints\\
  \texttt{e} \textit{opcode}&\textbf{E}xecute instruction (\textit{opcode})\\
  \texttt{i}&\textbf{I}data dump\\
  \texttt{f}&S\textbf{F}R dump\\
  \texttt{x} \textit{n}&\textbf{X}data dump (\textit{n}-th page)
\end{tabular}

\end{document}
